\begin{equation}
    \vec{R} = [n_{1}, n_{2}, n_{3}] = n_{1}\vec{a}_{1} + n_{2}\vec{a}_{2} + n_{3}\vec{a}_{3}
\end{equation}

\begin{equation}
    \vec{G} = h\vec{b}_{1} + k\vec{b}_{2} + l\vec{b}_{3}
\end{equation}

\begin{equation}
    d = \frac{2\pi}{|\vec{G}_{\text{min}}|}
\end{equation}

\begin{equation}
    \vec{b}_{i} \cdot \vec{a}_{j} = 2\pi \delta_{ij}
\end{equation}

\begin{equation}
    (hkl) \rightarrow_{\perp} h\vec{b}_{1} + k\vec{b}_{2} + l\vec{b}_{3}
\end{equation}

\begin{equation}
    2 d \sin(\theta) = n \lambda \comment{Bragg condition}
\end{equation}

\begin{equation}
    f(E,T) = \frac{1}{\exp(\frac{E-\mu}{K_{B}T}) + 1} \comment{Fermi-Dirac}
\end{equation}

\begin{equation}
    E_{f} = \frac{\hbar^{2}}{2m} (3\pi^{2}n)^{2/3} \comment{for free electron gas}
\end{equation}




